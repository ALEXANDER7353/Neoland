1. Inicializa tu proyecto
Crea una carpeta para tu proyecto:
bash

mkdir my-express-app
cd my-express-app

2. Inicializa npm:
npm init -y

3.Instala las dependencias necesarias:

npm install express mongoose dotenv body-parser cors

2. Estructura básica del proyecto
Organiza tu proyecto en una estructura clara:

my-express-app/
├── models/
│   └── User.js
├── routes/
│   └── userRoutes.js
├── config/
│   └── db.js
├── .env
├── server.js
└── package.json



3. Configura la conexión a MongoDB
En el archivo config/db.js:

const mongoose = require('mongoose');

const connectDB = async () => {
  try {
    await mongoose.connect(process.env.MONGO_URI, {
      useNewUrlParser: true,
      useUnifiedTopology: true,
    });
    console.log('MongoDB Connected...');
  } catch (err) {
    console.error(err.message);
    process.exit(1);
  }
};

module.exports = connectDB;


En el archivo .env:

MONGO_URI=mongodb://localhost:27017/mydatabase
PORT=5000




Crear una aplicación con Express y MongoDB usando Mongoose implica varios pasos. Aquí tienes una guía paso a paso para que puedas implementarlo:

1. Inicializa tu proyecto
Crea una carpeta para tu proyecto:
bash
Copiar código
mkdir my-express-app
cd my-express-app
Inicializa npm:
bash
Copiar código
npm init -y
Instala las dependencias necesarias:
bash
Copiar código
npm install express mongoose dotenv body-parser cors
2. Estructura básica del proyecto
Organiza tu proyecto en una estructura clara:

lua
Copiar código
my-express-app/
├── models/
│   └── User.js
├── routes/
│   └── userRoutes.js
├── config/
│   └── db.js
├── .env
├── server.js
└── package.json
3. Configura la conexión a MongoDB
En el archivo config/db.js:

javascript
Copiar código
const mongoose = require('mongoose');

const connectDB = async () => {
  try {
    await mongoose.connect(process.env.MONGO_URI, {
      useNewUrlParser: true,
      useUnifiedTopology: true,
    });
    console.log('MongoDB Connected...');
  } catch (err) {
    console.error(err.message);
    process.exit(1);
  }
};

module.exports = connectDB;
En el archivo .env:

env
Copiar código
MONGO_URI=mongodb://localhost:27017/mydatabase
PORT=5000




4. Crea un modelo con Mongoose
En el archivo models/User.js:

const mongoose = require('mongoose');

const UserSchema = new mongoose.Schema({
  name: {
    type: String,
    required: true,
  },
  email: {
    type: String,
    required: true,
    unique: true,
  },
  password: {
    type: String,
    required: true,
  },
}, { timestamps: true });

module.exports = mongoose.model('User', UserSchema);




5. Crea rutas y lógica para interactuar con el modelo
En el archivo routes/userRoutes.js:

const express = require('express');
const User = require('../models/User');

const router = express.Router();

// Ruta para obtener todos los usuarios
router.get('/', async (req, res) => {
  try {
    const users = await User.find();
    res.json(users);
  } catch (err) {
    res.status(500).json({ message: err.message });
  }
});

// Ruta para crear un usuario
router.post('/', async (req, res) => {
  const { name, email, password } = req.body;

  try {
    const newUser = new User({ name, email, password });
    await newUser.save();
    res.status(201).json(newUser);
  } catch (err) {
    res.status(400).json({ message: err.message });
  }
});

// Ruta para eliminar un usuario
router.delete('/:id', async (req, res) => {
  try {
    await User.findByIdAndDelete(req.params.id);
    res.json({ message: 'Usuario eliminado' });
  } catch (err) {
    res.status(500).json({ message: err.message });
  }
});

module.exports = router;




6. Configura el servidor principal
En el archivo server.js:


const express = require('express');
const bodyParser = require('body-parser');
const cors = require('cors');
const dotenv = require('dotenv');
const connectDB = require('./config/db');
const userRoutes = require('./routes/userRoutes');

dotenv.config();
connectDB();

const app = express();
const PORT = process.env.PORT || 5000;

// Middlewares
app.use(cors());
app.use(bodyParser.json());

// Rutas
app.use('/api/users', userRoutes);

app.listen(PORT, () => console.log(`Server running on http://localhost:${PORT}`));




7. Ejecuta el proyecto
1.Asegúrate de que MongoDB esté corriendo en tu máquina:

mongod

2.Inicia el servidor:

node server.js


3.Accede a las rutas:
GET http://localhost:5000/api/users - Obtén todos los usuarios.
POST http://localhost:5000/api/users - Crea un nuevo usuario enviando un JSON en el cuerpo.







